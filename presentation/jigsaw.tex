\documentclass[aspectratio=169,x11names,11pt]{beamer}
\usetheme{Pittsburgh}
\usecolortheme{dove} % seagull, dove, seahorse
\useoutertheme{sidebar} % sidebar [subsection=true]
\useinnertheme{inmargin}
\usepackage{xcolor}
\usepackage[utf8]{inputenc}
\usepackage[T1]{fontenc}
\usepackage[ngerman]{babel}
\usepackage{amsmath}
\usepackage{amsfonts}
\usepackage{amssymb}
\usepackage{graphicx}
\usepackage{multicol}
\usepackage{wrapfig}
\usepackage{lmodern}
\usepackage{listings}
\usepackage[autostyle=true,german=quotes]{csquotes}
\usepackage{hyperref}

\hypersetup{pdfstartview={Fit}}

\selectlanguage{ngerman}

\author[]{Henning Hontheim}
\title[]{Modularisierung in Java mit Jigsaw}
\date{}
% \date{\today}
% \logo{\includegraphics[height=.35cm]{images/logo_mono.eps}}

\setbeamercovered{invisible}
\beamertemplatenavigationsymbolsempty

% For Footnotes without markers on the slide
% https://tex.stackexchange.com/questions/30720/footnote-without-a-marker
\newcommand\blfootnote[1]{%
	\begingroup
	\renewcommand\thefootnote{}\footnote{#1}%
	\addtocounter{footnote}{-1}%
	\endgroup
}

\definecolor{cLightPlain}{HTML}{000000}
\definecolor{cLightKeyword}{HTML}{9B2393}
\definecolor{cLightComment}{HTML}{536579}
\definecolor{cLightString}{HTML}{C41A16}
\definecolor{cLightNumber}{HTML}{1C00CF}
\definecolor{cLightBackground}{HTML}{FFFFFF}

\definecolor{cDarkPlain}{HTML}{FFFFFF}
\definecolor{cDarkKeyword}{HTML}{FD8F3F}
\definecolor{cDarkComment}{HTML}{6C7986}
\definecolor{cDarkString}{HTML}{FC6A5D}
\definecolor{cDarkNumber}{HTML}{9686F5}
\definecolor{cDarkBackground}{HTML}{1F1F24}

\lstdefinestyle{codeStyleLight}{
	numbers=left,
	numbersep=5pt,
	breaklines=true,
	showstringspaces=false,
	frame=l ,
	xleftmargin=15pt,
	xrightmargin=15pt,
	basicstyle=\ttfamily\scriptsize\color{cLightPlain},
	stepnumber=1,
	keywordstyle=\color{cLightKeyword},
	commentstyle=\color{cLightComment},
	stringstyle=\color{cLightString},
	numberstyle=\tiny\color{cLightNumber},
	backgroundcolor=\color{cLightBackground}
}

\lstdefinestyle{codeStyleDark}{
	numbers=left,
	numbersep=5pt,
	breaklines=true,
	showstringspaces=false,
	frame=l ,
	xleftmargin=15pt,
	xrightmargin=15pt,
	basicstyle=\ttfamily\scriptsize\color{cDarkPlain},
	stepnumber=1,
	keywordstyle=\color{cDarkKeyword},
	commentstyle=\color{cDarkComment},
	stringstyle=\color{cDarkString},
	numberstyle=\tiny\color{cDarkBackground},
	backgroundcolor=\color{cDarkBackground}
}

\lstdefinelanguage{Jigsaw}{
	morekeywords={
		module,
		requires,
		exports,
		opens,
		transitive,
		to
	},
	sensitive=false, % keywords are not case-sensitive
	morecomment=[l]{//}, % l is for line comment
	morecomment=[s]{/*}{*/}, % s is for start and end delimiter
	morestring=[b]" % defines that strings are enclosed in double quotes
}

\lstset{style=codeStyleLight}

\begin{document}

\begin{frame}[plain]
\section*{Intro}
\vskip.35em
\begin{beamercolorbox}{title}
	\usebeamerfont{title}\inserttitle
\end{beamercolorbox}
\vskip5.825em
\begin{beamercolorbox}{author}
	\usebeamerfont{author}\insertauthor
\end{beamercolorbox}
\vskip.825em
\begin{beamercolorbox}{date}
	\usebeamerfont{date}\insertdate
\end{beamercolorbox}
\end{frame}

\begin{frame}[plain]{Inhalt}
\section*{Inhalt}
	\tableofcontents
\end{frame}

\begin{frame}{Was ändert sich mit Java 9?}
\section{Java 9$+$}
\begin{itemize}
	\item Verzeichnisstruktur hat sich geändert
	\item JRE nicht mehr in eigenem Verzeichnis des JDK
	\item Java-Libs rt.jar und tools.jar gibt es nicht mehr
	\item Einige interne APIs sind nicht mehr zugreifbar (z.B. einige \texttt{com.sun.*}- und \texttt{sun.misc.*}-Packages)
	\item Reflection ist eingeschränkt
	\item Linker \texttt{jlink}, der Distributionen erstellt, die ohne zusätzliches Java ausführbar sind
\end{itemize}
\end{frame}

\begin{frame}{Jigsaw}{Das Java Platform Module System (JPMS)}
\section{Jigsaw}
\begin{itemize}
	\item Skalierbarkeit von Java (z. B. für IoT)
	\item Erhöhung der Sicherheit und Wartbarkeit von Java
	\item Verbesserte Performance
	\item Förderung gut konstruierter und wartbarer Bibliotheken und Anwendungen
	\item \textbf{Explizite Abhängigkeitskontrolle} (Definition lose gekoppelter Services und zuverlässige Konfiguration)
	\item Einfache Handhabbarkeit
\end{itemize}
\hrule
\begin{enumerate}
	\item Die Modularisierung von Java selbst (mit \enquote{Platform Modules})
	\item Die Modularisierung von Bibliotheken und Anwendungen (mit \enquote{Application Modules})
\end{enumerate}
\end{frame}

\begin{frame}{Unterschiede zu OSGi}
\subsection{OSGi}
\begin{itemize}
	\item OSGi unterstützt Versionierung
	\item OSGi unterstützt \textit{Hot Deployment}, also Austausch der Module zur Laufzeit
\end{itemize}
\textbf{Jigsaw unterstützt beides nicht!}
\end{frame}

\begin{frame}{Verschiedene Arten von Modulen}
\section{Modularten}
\begin{itemize}
	\item \textbf{Platform Explicit Modules}: \textit{Platform Modules}\\
	Module der Java Runtime (siehe \texttt{\$ java \text{-}\text{-}list-modules})
	\item \textbf{Application Explicit Modules}\\
	JARs, die einen \textit{Module Descriptor} (\texttt{module-info.class}) enthalten
	\item \textbf{Open Modules}\\
	Zusätzlich zu \textit{Application Explicit Modules}: Export der Module zur Laufzeit für \textit{Deep Reflection}
	\item \textbf{Automatic Modules}\\
	JARs, die keinen \textit{Module Descriptor} enthalten
	\item \textbf{Unnamed Modules}\\
	Alle JARs des \texttt{CLASSPATH} werden zu genau einem (!) unbenannten Modul zusammengesasst. Diese werden (selbst mit \textit{Module Descriptor}) als \textit{Nicht-Jigsaw-Module} behandelt.
\end{itemize}
\end{frame}

\begin{frame}{Die \texttt{module-info.java}-Datei}
\subsection{\texttt{module-info.java}}
\lstinputlisting[language=Jigsaw]{./snippets/module-info.java}
\end{frame}

\newcommand{\bspeins}{Manuelle Verlinkung über den \texttt{MODULEPATH}}
\newcommand{\bspzwei}{Manuelle Verlinkung als einzelne JAR Module}

\begin{frame}{Beispiele}
\section{Beispiele}
\begin{enumerate}
	\item \hyperref[bsp1]{\bspeins}
	\item \hyperref[bsp2]{\bspzwei}
\end{enumerate}
\end{frame}

\begin{frame}{\bspeins}
\subsection{\texttt{MODULEPATH}}
\label{bsp1}
\lstinputlisting[language=sh,xleftmargin=0pt,xrightmargin =0pt]{./snippets/compile1.sh}
\end{frame}

\begin{frame}{\bspzwei}
\subsection{JAR Module}
\label{bsp2}
\lstinputlisting[language=sh,xleftmargin=0pt,xrightmargin =0pt]{./snippets/compile2.sh}
\end{frame}

\begin{frame}[fragile]{\bspzwei}{Analyse der Dependencies mit \texttt{jdeps}}
\begin{lstlisting}[language=sh,xleftmargin=0pt,xrightmargin=0pt]
$ jdeps -s --module-path modules modules/*.jar
net.hontheim.jigsaw.meineapp -> java.base
net.hontheim.jigsaw.meineapp -> net.hontheim.jigsaw.meinutil
net.hontheim.jigsaw.meinutil -> java.base
\end{lstlisting}
\end{frame}

\begin{frame}{Weitere Informationen}
\section{Weiterführendes}
Das Java Platform Module System (JPMS, "Jigsaw")\\
\url{https://www.torsten-horn.de/techdocs/Jigsaw.html}
\end{frame}

%\begin{frame}{Quellen}
%\section{Quellen}
%\bibliography{sources} 
%\bibliographystyle{plain}
%\end{frame}

\end{document}